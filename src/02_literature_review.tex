\chapter{Literature Review}

\section{Introduction}

This chapter reviews existing literature across three interconnected areas: \textbf{impact measurement and management (IMM)}, \textbf{public sector innovation (PSI)}, and the application of \textbf{artificial intelligence (AI)} in these domains. The goal is to establish a conceptual foundation for AI-supported, values-driven impact evaluation in public innovation ecosystems.

\section{Impact Measurement and Management (IMM)}

The measurement of impact, especially within social or public contexts, has evolved significantly in recent years. Scholars such as \textcite{ebrahim2014measuring} emphasize the need to align measurement approaches with a theory of change and organizational strategy. They argue that organizations often struggle to balance accountability and learning, especially when impact is diffuse or long-term.

Nicholls et al. (2012) highlight the tensions between standardized, quantitative measurement systems and the contextual, qualitative nature of many social interventions. Their work in the field of social entrepreneurship has helped formalize a typology of impact logic models, demonstrating that one-size-fits-all approaches rarely succeed.

In the German context, actors like Phineo and UnternehmerTUM have offered practical IMM frameworks tailored to social enterprises and innovation labs. These models combine stakeholder mapping, output-outcome mapping, and logic modelling to foster clarity in how public interventions are expected to generate value.

\section{Public Sector Innovation and Value Creation}

Public sector innovation requires institutions not only to introduce new tools or practices, but also to foster new forms of legitimacy, collaboration, and accountability \parencite{sun2019algorithmic}. The OECD has extensively documented the challenges and opportunities that come with innovation in government, including a growing emphasis on public value creation, citizen co-production, and agile experimentation \parencite{oecd2020publicsector}.

Wirtz et al. (2020) provide a conceptual model for the digital transformation of public services, noting that data-driven approaches can both enhance and erode trust depending on their transparency, inclusiveness, and perceived fairness.

The concept of \textbf{public value}—first developed by Moore (1995) and later expanded in various governance frameworks—has become a central reference point for assessing the outcomes of innovation. This aligns with the mission of the Public Value Hub and serves as a guiding principle in Project Athena.

\section{Artificial Intelligence in Evaluation and Decision-Making}

Artificial intelligence has seen increased use in public governance contexts—from algorithmic decision-making to NLP-based policy analysis. Devlin et al. (2018) introduced BERT, a transformer-based model for language understanding that has since become foundational for text classification, topic modeling, and semantic similarity tasks—tools that can be applied in impact evaluation scenarios.

Sun and Medaglia (2019) caution that AI must be embedded in deliberative and adaptive governance systems to ensure that its use enhances, rather than replaces, human judgment.

Recent work by Brown et al. (2020) explores how algorithmic systems can improve accountability by automating monitoring, but also warns of risks such as value misalignment, opacity, and exclusion.

In the context of Project Athena, AI is not used to make decisions, but to support them—offering new tools for extracting meaning from qualitative data, clustering impact narratives, and helping stakeholders explore the space of plausible outcomes.

\section{Synthesis and Gaps}

Despite growing attention to IMM, public innovation, and AI-supported decision-making, an integrated framework that combines all three is still underdeveloped. Current IMM approaches often lack the technical capacity to process unstructured data, while AI applications in public sector evaluation rarely account for the complexity of social outcomes or the normative commitments of the public sphere.

This thesis contributes to closing that gap by offering a use case where AI supports, rather than displaces, human interpretation—grounded in stakeholder input, aligned with public value frameworks, and informed by practical IMM processes from trusted intermediaries such as Phineo and UnternehmerTUM.

\section{Conclusion and Research Direction}

The review of current literature reveals that while significant progress has been made in IMM, public sector innovation, and AI applications individually, there remains a clear gap in integrating these domains into a cohesive framework. Existing IMM tools often fall short in processing qualitative, unstructured data, while many AI-driven evaluation systems overlook the normative dimensions of public value. This thesis responds to that gap by proposing a values-aligned, AI-supported approach to impact measurement, grounded in the lived experiences and expectations of public sector stakeholders.

To that end, the following chapter outlines the methodological strategy undertaken to bridge this gap. It combines qualitative insights from practitioners at the Public Value Hub with a practical AI implementation, demonstrating how intelligent systems can support — rather than replace — human judgment in the evaluation of public innovation initiatives.