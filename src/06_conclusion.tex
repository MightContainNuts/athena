\chapter{Conclusion}\label{ch:conclusion}

This chapter summarizes the key findings of the thesis, reflects on the theoretical, practical, and methodological contributions, and outlines directions for further research and development.

\section{Summary of Findings}\label{sec:summary-findings}

The thesis addressed the research question:

\begin{quote}
\textit{How can Artificial Intelligence support and improve Impact Measurement and Management in social enterprises through an artefact developed using the Design Science Research methodology?}
\end{quote}

Key insights include:

\begin{itemize}
    \item \textbf{AI-Supported IMM:} Natural language processing and semantic analysis can process both structured and unstructured impact data, bridging gaps in traditional IMM approaches.
    \item \textbf{Human-in-the-Loop Design:} Stakeholder validation is essential to maintain interpretability, legitimacy, and value alignment.
    \item \textbf{Artefact Validation:} The prototypical implementation in \textit{Inluma} demonstrated feasibility, transparency, and practical relevance for social enterprise impact evaluation.
    \item \textbf{Integration of Frameworks:} Combining IMM principles, AI methods, and public value considerations allows a more holistic evaluation of social innovation initiatives.
\end{itemize}

\section{Theoretical, Practical, and Methodological Contributions}\label{sec:contributions}

\textbf{Theoretical Contribution:}

\begin{itemize}
    \item Extends literature on AI-supported IMM by integrating qualitative and quantitative evaluation with human-in-the-loop processes.
    \item Provides a conceptual model linking AI, IMM frameworks, and public value for social enterprise contexts.
\end{itemize}

\textbf{Practical Contribution:}

\begin{itemize}
    \item Demonstrates a prototypical AI toolset capable of generating interpretable KPIs, clustering narrative inputs, and mapping initiatives to SDGs.
    \item Offers social enterprises a structured, semi-automated approach to enhance transparency, comparability, and evidence-based decision-making.
\end{itemize}

\textbf{Methodological Contribution:}

\begin{itemize}
    \item Shows how Design Science Research can be applied to develop, implement, and evaluate AI-supported artefacts in complex, value-driven domains.
    \item Highlights the importance of iterative, human-in-the-loop evaluation cycles for ensuring alignment with stakeholder needs.
\end{itemize}

\section{Limitations}\label{sec:limitations}

\begin{itemize}
    \item The artefact is prototypical and not intended as a market-ready product; scalability and longitudinal effects remain untested.
    \item Evaluation relied on synthetic and anonymized project data, as well as limited stakeholder walkthroughs.
    \item The approach is currently tailored to \textit{Inluma} and may require adaptation for other contexts or sectors.
\end{itemize}

\section{Outlook and Future Work}\label{sec:outlook}

Potential extensions include:

\begin{itemize}
    \item Integration with larger datasets and live project pipelines to evaluate longitudinal impact.
    \item Expansion of AI capabilities for more nuanced qualitative analysis, including sentiment, narrative trajectory, and stakeholder preference modeling.
    \item Adaptation of the framework for broader application in social enterprises, public administration, and international development contexts.
    \item Continued refinement of human-in-the-loop workflows to balance automation with ethical, transparent, and participatory decision-making.
\end{itemize}

\section{Closing Remarks}\label{sec:closing-remarks}

The thesis demonstrates that AI can augment human judgment in Impact Measurement and Management, providing actionable insights while preserving interpretability and ethical oversight. 
By combining IMM frameworks, AI methods, and public value considerations, the proposed artefact offers a pathway toward more transparent, systematic, and stakeholder-aligned evaluation of social innovation initiatives.