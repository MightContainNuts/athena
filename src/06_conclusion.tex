\chapter{Conclusion}\label{ch:conclusion}

This chapter summarizes the key findings of the thesis, reflects on the theoretical, practical, and methodological contributions, and outlines directions for further research and development.

\section{Summary of Findings}\label{sec:summary-findings}

The thesis addressed the research question:

\begin{quote}
\textit{How can Artificial Intelligence support and improve Impact Measurement and Management in social enterprises and public sector innovation contexts through an artefact developed using the Design Science Research methodology?}
\end{quote}

The study demonstrates that AI can meaningfully support Impact Measurement and Management when embedded within a transparent, human-in-the-loop design. Key insights include:

\begin{itemize}
    \item \textbf{AI-Supported IMM:} Natural language processing and semantic analysis enable the structured use of both qualitative and quantitative impact data, addressing limitations of indicator-driven IMM approaches.
    \item \textbf{Human-in-the-Loop Design:} Continuous stakeholder validation is essential to maintain interpretability, legitimacy, and alignment with public value and social impact objectives.
    \item \textbf{Artefact Validation:} The prototypical implementation within \textit{Inluma} demonstrated feasibility, transparency, and practical relevance according to Design Science evaluation criteria.
    \item \textbf{Integration of Frameworks:} Combining IMM principles, AI methods, and public value considerations supports a more holistic and reflective evaluation of social innovation initiatives.
\end{itemize}

\section{Theoretical, Practical, and Methodological Contributions}\label{sec:contributions}

\textbf{Theoretical Contribution:}

\begin{itemize}
    \item Extends existing work on AI-supported IMM by illustrating how qualitative narratives and quantitative indicators can be integrated through AI-assisted, human-in-the-loop processes.
    \item Contributes design knowledge on how AI, IMM frameworks, and public value concepts can be coherently linked in social enterprise and public innovation settings.
\end{itemize}

\textbf{Practical Contribution:}

\begin{itemize}
    \item Demonstrates a prototypical AI-enabled artefact capable of generating interpretable KPIs, clustering narrative inputs, and mapping initiatives to SDGs.
    \item Provides social enterprises and innovation intermediaries with a structured, semi-automated approach to enhance transparency, comparability, and evidence-informed decision-making.
\end{itemize}

\textbf{Methodological Contribution:}

\begin{itemize}
    \item Shows how Design Science Research can be applied to the development and evaluation of AI-supported artefacts in complex, value-driven domains.
    \item Highlights the importance of iterative evaluation cycles and human oversight in ensuring relevance and ethical alignment.
\end{itemize}

\section{Limitations}\label{sec:limitations}

\begin{itemize}
    \item The artefact is prototypical and not intended as a market-ready system; scalability, robustness, and long-term effects remain untested.
    \item Evaluation relied on synthetic and anonymized project data as well as a limited number of stakeholder walkthroughs.
    \item The current implementation is tailored to the \textit{Inluma} context and may require adaptation for other organizational or sectoral settings.
\end{itemize}

\section{Outlook and Future Work}\label{sec:outlook}

Future research and development may include:

\begin{itemize}
    \item Integration with larger datasets and live project pipelines to assess longitudinal impact development.
    \item Extension of AI capabilities for deeper qualitative analysis, such as narrative change over time, sentiment dynamics, or stakeholder perspective modeling.
    \item Adaptation of the artefact for broader application in public administration, social entrepreneurship ecosystems, and international development contexts.
    \item Further refinement of human-in-the-loop workflows to balance automation, transparency, and participatory decision-making.
\end{itemize}

\section{Closing Remarks}\label{sec:closing-remarks}

This thesis demonstrates that AI can act as a supportive cognitive tool in Impact Measurement and Management, enhancing sense-making and comparability while preserving human judgment and ethical oversight.
By integrating IMM frameworks, AI methods, and public value considerations, the proposed artefact offers a practical and reflective approach to evaluating social innovation initiatives in complex public and social contexts.