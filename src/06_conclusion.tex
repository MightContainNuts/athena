%! Author = deandidion
%! Date = 09.07.25

% Preamble
\chapter{Conclusion}\label{ch:conclusion}


\section{Summary}\label{sec:summary}

This thesis has investigated the potential and current applications of Artificial Intelligence (AI) in the field of impact measurement.
The research explored how AI techniques—particularly those in natural language processing—can support the interpretation of qualitative data and improve the overall scalability, consistency, and speed of impact assessments.

Key findings from interviews and case studies revealed a growing interest in AI-enabled tools among practitioners.
However, practical implementation remains limited due to challenges such as data fragmentation, lack of standardized frameworks, and ethical concerns surrounding AI opacity and bias.

The study also highlighted a consensus on the need for human-in-the-loop approaches, where AI serves as an assistive technology rather than a replacement.
This hybrid approach respects the complexity of social impact work and acknowledges the contextual knowledge of domain experts.

\section{Future Work}\label{sec:future-work}

Building on the insights gained through this study, several directions for future research and development are recommended:

\begin{itemize}
    \item \textbf{Standardization:} Promote the creation of open, interoperable data schemas and metrics that can be used across projects and platforms to improve comparability and reusability.
    \item \textbf{Tool Development:} Design and test AI tools tailored specifically to the needs of impact practitioners, including transparent summarization engines and ethical classifiers.
    \item \textbf{Human-AI Collaboration:} Investigate frameworks for participatory tool design and evaluation that embed practitioners, beneficiaries, and decision-makers in the development cycle.
    \item \textbf{Governance and Trust:} Examine regulatory models and trust-building mechanisms to ensure responsible AI use, particularly in high-stakes social environments.
    \item \textbf{Cross-Sector Learning:} Facilitate knowledge exchange between AI researchers, evaluators, social entrepreneurs, and funders to accelerate innovation and avoid siloed progress.
\end{itemize}

In sum, while the integration of AI into impact measurement is still in its early stages, the potential is significant.
With intentional design, collaboration, and a focus on responsible use, AI can help build more adaptive, transparent, and effective systems for understanding and improving social value.[16]