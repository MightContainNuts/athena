%! Author = deandidion
%! Date = 11.07.25

\chapter{Additional Data}\label{ch:additional-data}

This appendix contains supplementary materials that support the AI-supported IMM artefact described in Chapters~\ref{ch:artefact-development} and \ref{ch:demonstration-evaluation}.
It includes synthetic project profile examples, generated KPIs, audit log excerpts, and dashboard mock-ups.

\section*{Example Project Profiles}

Below are structured outputs from the \texttt{Profile} Pydantic model for three synthetic sample projects.
These illustrate the type of information extracted from pitch decks and structured for KPI generation.

\noindent\textit{The following profiles are synthetic examples generated for demonstration purposes:}

\begin{verbatim}
{
  "startup_name": "GreenFields AgTech",
  "legal_form": "GmbH",
  "founder_first_name": "Anna",
  "founder_last_name": "Schmidt",
  "founder_gender": "female",
  "startup_email": "anna@greenfields.com",
  "startup_phone": "+49 123 456 789",
  "startup_city": "Berlin",
  "startup_country": "Germany",
  "startup_postcode": "10115",
  "website": "https://greenfields.com",
  "project_beginning": "2023-03-01",
  "turnover": 250000,
  "profit": 30000,
  "employers": 5,
  "problem": "Excessive synthetic nitrogen usage in small farms",
  "vision": "Reduce fertilizer use while maintaining yields",
  "mission": "Develop sustainable precision farming tools",
  "solution": "IoT soil sensors with AI-driven recommendations",
  "social_impact": "Promote sustainable agriculture and environmental health",
  "reason": "Reduce environmental pollution and farmer costs",
  "value_1": "Sustainability",
  "value_2": "Efficiency",
  "value_3": "Innovation",
  "target_group": "Smallholder farmers in Europe"
}
\end{verbatim}

\begin{verbatim}
{
  "startup_name": "CareConnect Health",
  "legal_form": "UG",
  "founder_first_name": "David",
  "founder_last_name": "Müller",
  "founder_gender": "male",
  "startup_email": "contact@careconnect.health",
  "startup_phone": "+49 987 654 321",
  "startup_city": "Hamburg",
  "startup_country": "Germany",
  "startup_postcode": "20095",
  "website": "https://careconnect.health",
  "project_beginning": "2022-10-15",
  "turnover": 180000,
  "profit": -20000,
  "employers": 4,
  "problem": "Limited access to mental health support for young adults",
  "vision": "Affordable mental health support for everyone",
  "mission": "Deliver low-threshold digital mental health services",
  "solution": "Mobile app offering guided self-help and remote coaching",
  "social_impact": "Improved mental well-being and early intervention",
  "reason": "Address growing unmet mental health needs",
  "value_1": "Accessibility",
  "value_2": "Empathy",
  "value_3": "Trust",
  "target_group": "Young adults aged 18–30"
}
\end{verbatim}


\begin{verbatim}
{
  "startup_name": "CarbonSight",
  "legal_form": "GmbH",
  "founder_first_name": "Sofia",
  "founder_last_name": "Lindner",
  "founder_gender": "female",
  "startup_email": "hello@carbonsight.io",
  "startup_phone": "+49 555 123 987",
  "startup_city": "Munich",
  "startup_country": "Germany",
  "startup_postcode": "80331",
  "website": "https://carbonsight.io",
  "project_beginning": "2021-06-01",
  "turnover": 520000,
  "profit": 85000,
  "employers": 9,
  "problem": "SMEs lack tools to measure Scope 1 and 2 emissions",
  "vision": "Transparent emissions data for every business",
  "mission": "Enable data-driven climate action for SMEs",
  "solution": "Carbon accounting software with automated data ingestion",
  "social_impact": "Supports emission reductions and climate reporting",
  "reason": "Regulatory pressure and sustainability demand",
  "value_1": "Transparency",
  "value_2": "Accuracy",
  "value_3": "Responsibility",
  "target_group": "European SMEs"
}
\end{verbatim}


\section*{Generated KPIs / Indicators}

Example KPI generated from the above project profiles:

\begin{verbatim}
{
  "category": "Agrar & Agrar Tech",
  "sub_category": "Sustainable Inputs",
  "goal": ["Reduce synthetic nitrogen application rate by 20 percent per hectare within 12 months among users"],
  "short_term_goal_1": "Users reduce synthetic nitrogen rate within 6 months",
  "short_term_indicator_1": "Share of users who reduced synthetic nitrogen rate compared to last season",
  "short_term_question_1": "Did you reduce your synthetic nitrogen application rate per hectare this season compared to last season?",
  "type_of_short_term_question_1": "single_choice",
  "answer_options_short_term_question_1": ["Yes", "No", "Not applicable"],
  "measurement_method_short_term_question_1": "Self-reported comparison to baseline season recorded at onboarding",
  "unit_method_short_term_question_1": "percent of users",
  "justification_method_short_term_question_1": "User-level rate reduction is an early signal of improved nutrient efficiency",
  "source_method_short_term_question_1": "IRIS+ Agrochemical Use intensity; SDG 2.4",
  "long_term_goal_1": ["Users sustain a 20 percent lower synthetic nitrogen rate after 3 seasons"],
  "long_term_indicator_1": ["Kilograms of synthetic nitrogen applied per hectare per season"],
  "long_term_question_1": "How many kilograms of synthetic nitrogen per hectare did you apply this season?",
  "type_of_long_term_question_1": "open_question",
  "measurement_method_long_term_question_1": "Farmer input logs normalized by field area",
  "unit_method_long_term_question_1": "kg N/ha",
  "justification_method_long_term_question_1": "Rate per area directly measures fertilizer pressure with strong links to cost and emissions",
  "source_method_long_term_question_1": "IRIS+ Agrochemical Use; FAO fertilizer statistics; SDG 12.4"
}
\end{verbatim}

\textit{The following excerpt illustrates how multiple outcomes are detected and represented; non-essential fields are omitted for brevity.}

\begin{verbatim}
{
  "goal": [
    "Increase active platform usage",
    "Improve reporting consistency among startups"
  ],
  "short_term_goal_1": "Startups log in monthly",
  "short_term_goal_2": "Startups submit at least one KPI report per quarter",
  "justification": "Multiple outcomes detected: adoption and learning behaviour"
}
\end{verbatim}

\section*{Human-in-the-Loop Audit Logs}

\begin{quote}
\textit{"Short-term indicator was slightly ambiguous; refined wording to ensure farmers understand units and target."}
\end{quote}

\begin{quote}
\textit{"SDG mapping verified: matches SDG 2.4 (Zero Hunger) and SDG 12.4 (Responsible Consumption)"} 
\end{quote}

\begin{table}[H]
\centering
\caption{Example anonymized audit log for KPI refinement}
\begin{tabular}{p{0.18\textwidth} p{0.25\textwidth} p{0.45\textwidth}}
\toprule
\textbf{Stage} & \textbf{Actor} & \textbf{Comment / Change} \\
\midrule
Initial generation & LLM & KPI proposed with generic wording \\
Expert review & Domain expert & Clarified indicator unit and timeframe \\
SDG validation & Impact analyst & Confirmed SDG 2.4 and SDG 12.4 alignment \\
Final approval & Platform admin & KPI approved for deployment \\
\bottomrule
\end{tabular}
\end{table}

\section*{Dashboard Mock-Up}


The dashboard mock-up is populated with simulated KPI values based on realistic adoption and impact trajectories.
It visualizes KPI performance over time, compares projects within a cohort, and highlights contributions to public value dimensions such as sustainability, equity and participation.

\begin{figure}[H]
    \centering
    \includegraphics[width=0.8\textwidth]{../fig/impact_dashboard}
    \caption{Prototype Impact Dashboard Showing KPI Performance and Trends}
    \label{fig:impact_dashboard}
\end{figure}


\section*{Data Collection Instruments}

Example survey question derived from KPI:

\begin{quote}
\emph{Short-term KPI: Share of users who reduced synthetic nitrogen rate}  
\textbf{Question:} Did you reduce your synthetic nitrogen application rate per hectare this season compared to last season?  
\textbf{Type:} Single choice  
\textbf{Answer Options:} Yes / No / Not applicable  
\textbf{Measurement Method:} Self-reported comparison to baseline season recorded at onboarding
\end{quote}

\textbf{Long-term KPI Question:}
How many kilograms of synthetic nitrogen per hectare did you apply during the last growing season?


\textbf{Open-ended Question:}
How has the project influenced your understanding or behaviour regarding sustainable farming practices?

\section*{Raw Analysis Outputs}

Example aggregated results include mean KPI values per cohort, standard deviations, and trend indicators (e.g. improvement vs.\ baseline).
Qualitative responses are clustered thematically (e.g.\ transparency, usability, trust) using embedding-based methods and manually validated.



\section*{Ethical and Governance Documentation}

The AI-supported IMM artefact is designed and evaluated in accordance with ethical, legal, and governance principles, with a particular focus on data protection, transparency, and human oversight.

\begin{itemize}

    \item \textbf{GDPR-compliant consent form template (redacted):}
    All participants whose data is processed within the artefact provide informed consent prior to data collection.
    The consent form clearly specifies the purpose of data processing, the types of data collected (e.g.\ survey responses, usage metrics), storage duration, and participants’ rights (access, rectification, deletion).
    Personal identifiers are collected only where strictly necessary and are excluded from analytical outputs.
    A redacted version of the consent form is used in this appendix to avoid disclosure of sensitive or identifying information.

    \item \textbf{Anonymization and secure storage procedures:}
    Collected data is anonymized at the earliest possible stage. Direct identifiers (e.g.\ names, email addresses, phone numbers) are removed or replaced by pseudonymous IDs.
    Analytical datasets operate exclusively on aggregated or pseudonymized data at project or cohort level.
    All data is stored on secure servers with access control, and transmission is encrypted using standard protocols.
    Raw data access is restricted to authorized personnel involved in evaluation and quality assurance.
    Access to datasets, audit logs, and dashboard views is role-based, ensuring that only authorized analysts or administrators can access sensitive information.

    \item \textbf{Guidelines for human-in-the-loop oversight:}
    Human oversight is enforced at all critical stages of the IMM pipeline.
    AI-generated outputs (e.g.\ extracted profiles, generated KPIs, SDG mappings, narrative summaries) are treated as decision-support artifacts rather than final decisions.
    Domain experts review, edit, and approve outputs before they are used for assessment or reporting.
    All revisions and approvals are logged to ensure traceability and accountability.
    This human-in-the-loop approach mitigates risks of misinterpretation, bias, or inappropriate automation and aligns the artefact with responsible AI principles.

\end{itemize}

Together, these measures ensure that the artefact complies with GDPR requirements, supports transparency and explainability, and embeds ethical governance mechanisms throughout the AI-supported impact measurement process.

\section*{Glossary and Abbreviations}

\begin{itemize}
    \item KPI – Key Performance Indicator  
    \item IMM – Impact Measurement and Management  
    \item SDG – Sustainable Development Goal  
    \item LLM – Large Language Model  
    \item XAI – Explainable Artificial Intelligence  
    \item IRIS+ – Impact Reporting and Investment Standards
    \item LangGraph – Framework for orchestrating multi-step LLM workflows
    \item PyPDF – Python library for PDF parsing and text extraction
    \item Profile Pydantic Model – Structured schema used to store project onboarding data
    \item EPCIS – Electronic Product Code Information Services (traceability standard)
    \item GS1 Digital Link – Standard for resolving product identifiers to digital resources
\end{itemize}


\textbf{Note:} This appendix is intended to provide transparency and reproducibility for the artefact’s processing and outputs, while keeping all data anonymized and compliant with privacy standards.  