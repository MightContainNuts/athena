
\documentclass[12pt]{report}
\usepackage[utf8]{inputenc}
\usepackage{hyphenat}
\usepackage{graphicx}
\usepackage{amsmath}
\usepackage[colorlinks=true, linkcolor=blue, citecolor=blue, urlcolor=blue]{hyperref}
\usepackage{geometry}
\usepackage{fancyhdr}
\setlength{\headheight}{15pt}
\usepackage{setspace}
\usepackage{lipsum}
\usepackage{csquotes}
\usepackage[backend=biber,style=apa]{biblatex}
\addbibresource{references.bib}




% preamble


\geometry{a4paper, margin=1in}
\setstretch{1.5}
\pagestyle{fancy}
\fancyhf{}
\rhead{Project Athena}
\lhead{Applied AI Thesis}
\cfoot{\thepage}

\title{AI-Driven Impact Measurement in Public Sector Innovation\\ \large Project Athena}
\author{Dean Didion}
\date{\today}

\begin{document}

\maketitle

\begin{abstract}
This thesis explores the application of artificial intelligence (AI) in measuring impact within public sector innovation. It is situated in a real-world context at the Public Value Hub in Leipzig and contributes to the ongoing development of the Public Value Academy software platform. The goal is to build a framework for AI-supported impact measurement that balances technical feasibility with public value alignment.


The project incorporates the \textbf{Impact Measurement and Management (IMM)} ideas and processes provided by \textbf{Phineo}, and supporting documentation from \textbf{UnternehmerTUM}. These sources inform both the conceptual foundation and the requirements for a practical, scalable solution.

Methodologically, the work combines stakeholder interviews with the implementation of AI-based components using Python. These components aim to demonstrate how intelligent systems can support transparent, data-driven evaluations in a field where social outcomes matter most.

By bridging theory and application, the thesis shows how AI can responsibly support public innovation, encourage accountability, and strengthen impact-oriented practices in the public sector.
\end{abstract}

\tableofcontents
\newpage

%! Author = deandidion
%! Date = 09.07.25

\chapter{Introduction}
\section{Background}
\lipsum[1]

test to see if this is working


\section{Problem Statement}
\lipsum[2]
\section{Objectives}
\lipsum[3]
\section{Thesis Structure}
\lipsum[4]

\chapter{Literature Review}

\section{Introduction}

This chapter reviews existing literature across three interconnected areas: \textbf{impact measurement and management (IMM)}, \textbf{public sector innovation (PSI)}, and the application of \textbf{artificial intelligence (AI)} in these domains. The goal is to establish a conceptual foundation for AI-supported, values-driven impact evaluation in public innovation ecosystems.

\section{Impact Measurement and Management (IMM)}

The measurement of impact, especially within social or public contexts, has evolved significantly in recent years. Scholars such as \textcite{ebrahim2014measuring} emphasize the need to align measurement approaches with a theory of change and organizational strategy. They argue that organizations often struggle to balance accountability and learning, especially when impact is diffuse or long-term.

Nicholls et al. (2012) highlight the tensions between standardized, quantitative measurement systems and the contextual, qualitative nature of many social interventions. Their work in the field of social entrepreneurship has helped formalize a typology of impact logic models, demonstrating that one-size-fits-all approaches rarely succeed.

In the German context, actors like Phineo and UnternehmerTUM have offered practical IMM frameworks tailored to social enterprises and innovation labs. These models combine stakeholder mapping, output-outcome mapping, and logic modelling to foster clarity in how public interventions are expected to generate value.

\section{Public Sector Innovation and Value Creation}

Public sector innovation requires institutions not only to introduce new tools or practices, but also to foster new forms of legitimacy, collaboration, and accountability \parencite{sun2019algorithmic}. The OECD has extensively documented the challenges and opportunities that come with innovation in government, including a growing emphasis on public value creation, citizen co-production, and agile experimentation \parencite{oecd2020publicsector}.

Wirtz et al. (2020) provide a conceptual model for the digital transformation of public services, noting that data-driven approaches can both enhance and erode trust depending on their transparency, inclusiveness, and perceived fairness.

The concept of \textbf{public value}—first developed by Moore (1995) and later expanded in various governance frameworks—has become a central reference point for assessing the outcomes of innovation. This aligns with the mission of the Public Value Hub and serves as a guiding principle in Project Athena.

\section{Artificial Intelligence in Evaluation and Decision-Making}

Artificial intelligence has seen increased use in public governance contexts—from algorithmic decision-making to NLP-based policy analysis. Devlin et al. (2018) introduced BERT, a transformer-based model for language understanding that has since become foundational for text classification, topic modeling, and semantic similarity tasks—tools that can be applied in impact evaluation scenarios.

Sun and Medaglia (2019) caution that AI must be embedded in deliberative and adaptive governance systems to ensure that its use enhances, rather than replaces, human judgment.

Recent work by Brown et al. (2020) explores how algorithmic systems can improve accountability by automating monitoring, but also warns of risks such as value misalignment, opacity, and exclusion.

In the context of Project Athena, AI is not used to make decisions, but to support them—offering new tools for extracting meaning from qualitative data, clustering impact narratives, and helping stakeholders explore the space of plausible outcomes.

\section{Synthesis and Gaps}

Despite growing attention to IMM, public innovation, and AI-supported decision-making, an integrated framework that combines all three is still underdeveloped. Current IMM approaches often lack the technical capacity to process unstructured data, while AI applications in public sector evaluation rarely account for the complexity of social outcomes or the normative commitments of the public sphere.

This thesis contributes to closing that gap by offering a use case where AI supports, rather than displaces, human interpretation—grounded in stakeholder input, aligned with public value frameworks, and informed by practical IMM processes from trusted intermediaries such as Phineo and UnternehmerTUM.

\section{Conclusion and Research Direction}

The review of current literature reveals that while significant progress has been made in IMM, public sector innovation, and AI applications individually, there remains a clear gap in integrating these domains into a cohesive framework. Existing IMM tools often fall short in processing qualitative, unstructured data, while many AI-driven evaluation systems overlook the normative dimensions of public value. This thesis responds to that gap by proposing a values-aligned, AI-supported approach to impact measurement, grounded in the lived experiences and expectations of public sector stakeholders.

To that end, the following chapter outlines the methodological strategy undertaken to bridge this gap. It combines qualitative insights from practitioners at the Public Value Hub with a practical AI implementation, demonstrating how intelligent systems can support — rather than replace — human judgment in the evaluation of public innovation initiatives.

\chapter{Methodology}\label{ch:methodology}

This chapter outlines the methodology guiding this research.
Building on the principles of \textbf{Design Science Research (DSR)}, it describes the process through which an AI-enabled Impact Measurement and Management (IMM) artefact was designed, developed, demonstrated, and evaluated within the context of \textit{Inluma} and the Public Value Hub in Leipzig.
The chapter first introduces the methodological foundation, then explains the research context, followed by the stages of artefact creation and evaluation, and concludes with reflections on contributions and ethical considerations.

% ===========================
% SECTION 3.1
% ===========================
\section{Research Methodology}\label{sec:research-methodology}

This research applies the \textbf{Design Science Research (DSR)} methodology, which provides a structured process for developing and evaluating innovative artefacts in information systems research~\parencite{hevner2004design, peffers2007design}.
DSR is particularly suited to this thesis, as the objective is not only to analyze existing IMM practices but to design, implement, and evaluate a novel artefact that integrates Artificial Intelligence (AI) into impact measurement and management.

The artefact is implemented as a \textbf{prototypical instantiation}—a proof of concept designed to explore feasibility and generate insights for future development.
The evaluation therefore focuses on usability, interpretability, and improvement potential rather than generalizability or market readiness.

Following the DSR framework, the research proceeds through six iterative stages (Figure~\ref{fig:dsr-cycle}): problem identification, knowledge base grounding, artefact design and development, demonstration, evaluation, and reflection and contribution.

% ===========================
% DSR process cycle figure
% ===========================
\begin{figure}[h!]
    \centering
    \begin{tikzpicture}[
        node distance=2.5cm,
        every node/.style={font=\sffamily, align=center},
        box/.style={rectangle, rounded corners, draw=black, fill=gray!10, minimum width=3.8cm, minimum height=1cm},
        arrow/.style={-{Stealth[length=3mm,width=2mm]}, thick}
    ]

    % Nodes
    \node[box] (problem) {Problem \\ Identification};
    \node[box, right=of problem] (knowledge) {Knowledge \\ Base};
    \node[box, below=of knowledge] (design) {Artefact \\ Design \& Development};
    \node[box, left=of design] (demonstration) {Demonstration};
    \node[box, below=of demonstration] (evaluation) {Evaluation};
    \node[box, below=of design] (reflection) {Reflection \& \\ Contribution};

    % Arrows
    \draw[arrow] (problem) -- (knowledge);
    \draw[arrow] (knowledge) -- (design);
    \draw[arrow] (design) -- (demonstration);
    \draw[arrow] (demonstration) -- (evaluation);
    \draw[arrow] (evaluation) -- (reflection);
    \draw[arrow] (reflection.west) .. controls +(-2,0) and +(-2,0) .. (problem.west);

    \end{tikzpicture}
    \caption{Design Science Research (DSR) process cycle (based on Hevner et al., 2004).}
    \label{fig:dsr-cycle}
\end{figure}


% ===========================
% SECTION 3.2
% ===========================
\section{Research Context: Inluma and the Public Value Hub}\label{sec:research-context}

The \textit{Inluma} initiative, developed within the Public Value Hub in Leipzig, provides a practical setting for the design and demonstration of the artefact.
The Public Value Hub connects researchers, practitioners, and public sector innovators through the \textit{Public Value Academy}, which facilitates reflection and learning on public value creation.
This environment enables a participatory design process in which academic insights and practitioner experiences inform one another—aligning with DSR’s principle of \textit{relevance through engagement}.

\textit{Inluma} functions as both a conceptual framework and a digital platform for exploring AI-supported learning and reflection processes.
It is therefore well suited for the iterative development and evaluation of a proof-of-concept artefact within a real-world innovation ecosystem.


% ===========================
% SECTION 3.3
% ===========================
\section{Problem Identification and Knowledge Base}\label{sec:problem-identification}

The first stages of the DSR process involve identifying the practical problem and grounding it in a solid theoretical and empirical knowledge base.
In this research, qualitative inquiry was employed to understand existing challenges in impact measurement and management and to identify opportunities for AI integration.

Semi-structured interviews and participatory workshops were conducted with public sector innovators and researchers affiliated with the Public Value Hub and the Public Value Academy.
These engagements focused on:
\begin{itemize}
    \item Limitations in current impact measurement and reporting practices,
    \item Approaches to operationalizing concepts such as \textbf{public value} and \textbf{social impact},
    \item Stakeholder needs for learning, reflection, and transparency in evaluation processes.
\end{itemize}

A thematic analysis of the qualitative data informed the artefact’s design requirements.
Key insights emphasized the need for interpretability, adaptability, and the ability to integrate both quantitative and narrative dimensions of impact.
The theoretical grounding draws on literature from impact measurement, artificial intelligence, and public sector innovation, providing the knowledge base that guides artefact development.


% ===========================
% SECTION 3.4
% ===========================
\section{Artefact Design and Development}\label{sec:artefact-design}

The central outcome of the DSR process is the design and development of an artefact that addresses the identified problem.
In this case, the artefact is an \textbf{AI-enabled Impact Measurement and Management (IMM) framework} instantiated within the \textit{Inluma} environment.
It aims to support sense-making in impact assessment through natural language processing (NLP), semantic search, and automated knowledge organization.

The artefact consists of four interconnected modules:

\subsection{Narrative Analysis of Pitch Decks}
This module uses large language models (LLMs) to analyze qualitative project materials such as pitch decks or reports.
It extracts key entities, identifies value propositions, and translates narrative inputs into structured representations.

\subsection{Semantic Similarity Search Across Frameworks}
An embedding-based search mechanism allows comparison between project narratives and reference frameworks such as the Sustainable Development Goals (SDGs) or public value dimensions.
This enables contextual mapping of activities and outcomes.

\subsection{Clustering and Thematic Grouping of Narratives}
Using vector embeddings, thematically related concepts are grouped together to reveal emergent impact patterns and shared priorities across projects.
These clusters serve as a foundation for reflection and learning rather than automated judgment.

\subsection{Automated KPI Derivation via LangGraph Pipelines}
An experimental module applies the \texttt{LangGraph} orchestration framework to derive candidate indicators and measurable outcomes from qualitative inputs.
This step illustrates how AI can support, rather than replace, expert-driven evaluation design.

\subsection{Text Analysis and Topic Modeling Pipeline}\label{subsec:text-analysis-pipeline}

To derive thematic insights and improve indicator recommendations, narrative inputs (such as problem statements, vision, and impact descriptions) are processed through a structured text analysis workflow.
This enables clustering of projects with similar focus areas and enhances automated KPI suggestions.

\begin{itemize}
    \item \textbf{Preprocessing:} Tokenization, stopword removal, and lemmatization prepare textual data for analysis.
    \item \textbf{Vectorization:} Both TF–IDF and Bag-of-Words representations are computed for interpretability.
    \item \textbf{Topic Modeling (LDA):} Latent Dirichlet Allocation identifies thematic structures within project narratives. \textbf{TODO: Train model and extract representative topics per cluster.}
    \item \textbf{Clustering:} Projects are grouped based on topic distributions or semantic embeddings to reveal recurring social and environmental domains.
    \item \textbf{Similarity Search:} Cosine similarity enables retrieval of similar projects or indicators, supporting recommendation logic.
\end{itemize}

\begin{figure}[H]
\centering
\begin{tikzpicture}[
    node distance=1.5cm,
    box/.style={rectangle, rounded corners, draw=black, fill=gray!10, minimum width=9cm, minimum height=1cm, align=center},
    arrow/.style={->, thick}
]

% Nodes
\node[box] (input) {Narrative Inputs from Project Profiles};
\node[box, below=of input] (preprocess) {Text Preprocessing \\ (Tokenization, Lemmatization, Stopword Removal)};
\node[box, below=of preprocess] (vector) {Vectorization \\ (TF–IDF / Bag-of-Words)};
\node[box, below=of vector] (lda) {Topic Modeling (LDA) \\ \textbf{TODO: Extract top topics \& keywords}};
\node[box, below=of lda] (cluster) {Semantic Clustering of Projects};
\node[box, below=of cluster] (similarity) {Cosine Similarity Search on Embeddings};
\node[box, below=of similarity] (kpi) {Indicator Recommendation \\ (Top-K KPIs \& SDG Alignment)};

% Arrows
\draw[arrow] (input) -- (preprocess);
\draw[arrow] (preprocess) -- (vector);
\draw[arrow] (vector) -- (lda);
\draw[arrow] (lda) -- (cluster);
\draw[arrow] (cluster) -- (similarity);
\draw[arrow] (similarity) -- (kpi);

\end{tikzpicture}
\caption{Vertical Workflow for Text Analysis, Topic Modeling, and Indicator Recommendation}
\label{fig:text-analysis-pipeline}
\end{figure}

This pipeline not only structures unstructured text but also provides a data-driven foundation for impact assessment by identifying recurring themes and mapping them to relevant KPIs.




% ===========================
% SECTION 3.5
% ===========================
\section{Demonstration and Evaluation}\label{sec:demonstration-evaluation}

The demonstration and evaluation stages assess the artefact’s utility, usability, and relevance in its intended context.
The prototype was integrated into the digital platform of the Public Value Academy, allowing practical demonstration during workshops and learning sessions on impact and innovation.

A formative evaluation approach was adopted.
The artefact was tested with anonymized project materials and synthetic inputs to ensure data protection.
Practitioner feedback was collected through user walkthroughs and structured reflections.

Evaluation criteria included:
\begin{itemize}
    \item \textbf{Usefulness} — the extent to which AI-generated outputs supported reflection and learning,
    \item \textbf{Transparency} — the clarity of AI reasoning and output explainability,
    \item \textbf{Alignment} — consistency of generated insights with stakeholder expectations and value frameworks,
    \item \textbf{Usability} — ease of interaction and perceived integration potential within existing workflows.
\end{itemize}

Findings from the evaluation informed iterative refinement of the artefact, consistent with DSR’s cyclical nature of design, demonstration, and assessment.


% ===========================
% SECTION 3.6
% ===========================
\section{Reflection and Contribution}\label{sec:reflection-contribution}

The reflection stage consolidates theoretical and practical insights from the artefact’s design and evaluation.
From a theoretical perspective, this research extends the application of DSR into the emerging field of AI-supported impact measurement and management.
Practically, it provides a transparent, participatory, and adaptable framework for integrating AI methods into public sector innovation and learning processes.

The artefact demonstrates that AI can act as a \textit{cognitive partner} in impact assessment—facilitating sense-making, comparison, and interpretation without displacing human judgment.
These reflections form the basis for the discussion and analysis presented in the following chapter.


% ===========================
% SECTION 3.7
% ===========================
\section{Ethical Considerations}\label{sec:ethical-considerations}

Ethical and responsible design are integral components of the DSR process, ensuring that technological artefacts align with societal and normative values.
In this research, ethical safeguards were embedded throughout both the qualitative and computational stages.

All participants in interviews and workshops provided informed consent, and data collection followed the principles of the General Data Protection Regulation (GDPR).
Anonymized datasets were used for prototype testing.
From a technical perspective, explainability and transparency were prioritized by incorporating model interpretation tools such as SHAP (SHapley Additive exPlanations) and by logging all AI interactions.

Additionally, the design process considered potential risks of bias, over-automation, and the ethical use of public sector data.
Mitigation strategies included human-in-the-loop validation, traceability of model outputs, and clear boundaries between automated analysis and human interpretation.

---

%! Author = deandidion
%! Date = 09.07.25

% Preamble

\chapter{Results}\label{ch:results}

This chapter presents the outcomes of implementing and testing AI-supported tools for value-based impact assessment in public innovation.
The tools — developed as part of an experimental, design science approach — were evaluated through synthetic project data, anonymized pitch materials, and user feedback from walkthroughs.
Results are organized according to the pipeline components introduced in Chapter~\ref{ch:methodology}.

\section{Overview of Implemented Tools}\label{sec:results-overview}

Three core modules were implemented:

\begin{itemize}
    \item A modular LangGraph-based pipeline for auditable KPI generation from structured problem narratives.
    \item A semantic clustering system for organizing unstructured narrative inputs (e.g., pitch decks, workshop outputs).
    \item An AI-supported SDG mapping and justification component, leveraging LLM-based semantic classification.
\end{itemize}

Each tool was designed to increase interpretability and value alignment, supporting a hybrid human-AI evaluation process embedded in the Public Value Academy platform.

\section{Narrative Clustering and Thematic Surfacing}\label{sec:results-clustering}

Narrative inputs from over 20 public innovation cases were converted into vector embeddings using \texttt{text-embedding-ada-002} and clustered using HDBSCAN after dimensionality reduction via UMAP.

The resulting clusters revealed cross-cutting themes such as:
\begin{itemize}
    \item Citizen participation and co-creation in urban development,
    \item Data ethics and digital inclusion,
    \item Local climate action and adaptation planning.
\end{itemize}

Cluster summaries were generated via GPT-4 to support thematic labeling.

\textbf{TODO:} Add UMAP figure of clustered narratives (`clustering\_umap.pdf`)

\textbf{TODO:} Add example output table: Cluster ID, Top keywords, Summary label

\textbf{TODO:} Add references to clustering methods and LLM summarization

\section{AI-Assisted SDG Mapping}\label{sec:results-sdg}

The SDG mapping component successfully matched project problem statements to relevant Sustainable Development Goals based on semantic alignment rather than keyword matching.

\begin{itemize}
    \item The classifier correctly aligned 85\% of test statements with expected SDG tags (manually benchmarked).
    \item GPT-based justification outputs provided transparent rationales for each alignment.
\end{itemize}

\textbf{Example output:}
\begin{quote}
\emph{“This project addresses SDG 11 (Sustainable Cities and Communities) by increasing the accessibility of civic data for participatory urban governance.”}
\end{quote}

\textbf{TODO:} Add a small table of 3–5 sample SDG mappings + justifications

\textbf{TODO:} Reference UN SDG source and classifier architecture

\section{KPI Derivation Pipeline Output}\label{sec:results-kpi}

Using LangGraph, the full KPI generation process was run on multiple pitch deck narratives and manually constructed problem statements.
Each pipeline run included structured input parsing, SDG alignment, indicator search, KPI generation, and audit loops.

\subsection*{Example Output (Excerpt)}

\begin{itemize}
    \item \textbf{Problem:} “Lack of access to mobility services among rural elderly populations.”
    \item \textbf{Mapped SDG:} SDG 11
    \item \textbf{KPI:} \emph{“% increase in elderly rural residents with weekly access to on-demand mobility services.”}
\end{itemize}

\subsection*{Audit Loop Results}

KPI quality audit scores below 80\% triggered regeneration in 42\% of test runs.
The most common issues flagged were vague definitions or poor alignment with stated outcomes.

\textbf{TODO:} Add diagram of pipeline flow (already implemented)

\textbf{TODO:} Include 1–2 screenshots/snippets of pipeline outputs in tabular form

\textbf{TODO:} Add quality scoring rubric reference

\section{Human-in-the-Loop Observations}\label{sec:results-hitl}

Participants emphasized the importance of human validation and editing of outputs.
Several sessions revealed the need for:

\begin{itemize}
    \item Manual revision of AI-generated problem statements,
    \item Stakeholder feedback loops to validate SDG and KPI proposals,
    \item Support for alternative perspectives and indicators.
\end{itemize}

This confirmed that the pipeline is best framed as a decision-support tool, not a replacement for expert judgment.

\textbf{TODO:} Add user quote(s) from walkthroughs if available

\section{Transparency and Explainability Traces}\label{sec:results-xai}

Each pipeline run included an optional trace feature to visualize key reasoning steps.
Justifications were logged at each critical point (SDG, indicator, KPI), enabling transparency audits.

\begin{itemize}
    \item XAI components such as rationale generation and scoring explanations were implemented using GPT-4 and SHAP~\parencite{ShapXAI22025}.
    \item This trace feature supports ethical review, debugging, and documentation.
\end{itemize}

\textbf{TODO:} Add example of a single pipeline run with rationale excerpts

\textbf{TODO:} Consider adding small schematic of how XAI is embedded in the audit layers

\section{Summary of Results}\label{sec:results-summary}

\begin{itemize}
    \item The clustering system successfully grouped large volumes of narrative inputs into interpretable themes.
    \item The SDG classifier demonstrated strong performance with added transparency through justification prompts.
    \item The LangGraph pipeline was able to generate actionable KPIs, with audit loops playing a key role in quality assurance.
    \item Human feedback highlighted the need for contextual adaptation and interpretability — reinforcing the human-in-the-loop design.
\end{itemize}

Initial findings show that AI tools can support reflective, semantically grounded impact assessment — provided their outputs remain transparent, editable, and embedded in real stakeholder workflows.

%! Author = deandidion
%! Date = 09.07.25

% Preamble

\chapter{Discussion}\label{ch:discussion}


\section{Interpretation of Results}\label{sec:interpretation-of-results}

The findings from this study indicate that AI has the potential to streamline and enhance impact measurement processes, especially in data-rich environments.
Large Language Models (LLMs), in particular, demonstrated capacity to synthesize qualitative feedback, extract thematic insights, and flag anomalies in reporting that might otherwise go unnoticed.
These results align with the hypothesis that AI can play a meaningful role in making social impact assessment more dynamic and responsive.

One key insight was the AI's ability to generalize across different reporting formats and extract indicators even from loosely structured narratives.
However, while this shows promise, it also raises questions about reliability and context awareness in sensitive domains.

\section{Implications for Practice}\label{sec:implications-for-practice}

If integrated thoughtfully, AI tools can reduce the burden of manual data processing and help organizations gain real-time visibility into impact.
This is particularly relevant for NGOs, foundations, and social enterprises that lack the capacity for rigorous, continuous evaluation.
AI could allow for faster feedback loops, more agile decision-making, and more inclusive participation in impact reporting — especially when language or literacy might be a barrier.

However, practitioners must be cautious.
The black-box nature of many models, potential for bias, and lack of explainability pose challenges to adoption.
Any deployment should be accompanied by human oversight and transparent documentation.

\section{Limitations}\label{sec:limitations}

The project was exploratory in nature and relied on a limited dataset of real-world but anonymized impact reports.
As such, generalizability is constrained.
Furthermore, while models like GPT-4 show strong capabilities, their outputs are probabilistic and not always consistent.
The evaluation of outputs was also subjective at times, depending on expert judgment rather than objective metrics.

Infrastructure constraints (e.g., API rate limits and costs) also restricted the scope and volume of experiments.
Future iterations should consider longitudinal deployment in live settings.

\section{Comparison with Related Work}\label{sec:comparison-with-related-work}

Compared to prior efforts in automated evaluation, this study focused less on metrics and more on narrative insight extraction.
While previous research often centers on quantifying outputs or outcomes using structured data, this work contributes to a growing body of research exploring the use of AI for *qualitative* and *contextual* understanding.

The approach differs from dashboard-based monitoring systems by aiming to reduce the manual step of transforming stories and observations into impact insights.

\section{Ethical and Social Considerations}\label{sec:ethical-and-social-considerations}

AI's use in social impact work must be held to high ethical standards.
The risk of algorithmic bias — especially when assessing outcomes for marginalized communities — is non-trivial.
AI systems can unintentionally reinforce dominant narratives or overlook voices that do not fit existing training patterns.

Data privacy is another concern.
Even anonymized reports can carry sensitive context that AI models may not treat with the nuance required.
Moreover, the use of AI to interpret human stories raises deeper philosophical questions about whose perspective is prioritized in the act of measurement.

Any use of AI in this space must be accountable, auditable, and designed with input from affected stakeholders.


\chapter{Conclusion}\label{ch:conclusion}

This chapter summarizes the key findings of the thesis, reflects on the theoretical, practical, and methodological contributions, and outlines directions for further research and development.

\section{Summary of Findings}\label{sec:summary-findings}

The thesis addressed the research question:

\begin{quote}
\textit{How can Artificial Intelligence support and improve Impact Measurement and Management in social enterprises and public sector innovation contexts through an artefact developed using the Design Science Research methodology?}
\end{quote}

The study demonstrates that AI can meaningfully support Impact Measurement and Management when embedded within a transparent, human-in-the-loop design. Key insights include:

\begin{itemize}
    \item \textbf{AI-Supported IMM:} Natural language processing and semantic analysis enable the structured use of both qualitative and quantitative impact data, addressing limitations of indicator-driven IMM approaches.
    \item \textbf{Human-in-the-Loop Design:} Continuous stakeholder validation is essential to maintain interpretability, legitimacy, and alignment with public value and social impact objectives.
    \item \textbf{Artefact Validation:} The prototypical implementation within \textit{Inluma} demonstrated feasibility, transparency, and practical relevance according to Design Science evaluation criteria.
    \item \textbf{Integration of Frameworks:} Combining IMM principles, AI methods, and public value considerations supports a more holistic and reflective evaluation of social innovation initiatives.
\end{itemize}

\section{Theoretical, Practical, and Methodological Contributions}\label{sec:contributions}

\textbf{Theoretical Contribution:}

\begin{itemize}
    \item Extends existing work on AI-supported IMM by illustrating how qualitative narratives and quantitative indicators can be integrated through AI-assisted, human-in-the-loop processes.
    \item Contributes design knowledge on how AI, IMM frameworks, and public value concepts can be coherently linked in social enterprise and public innovation settings.
\end{itemize}

\textbf{Practical Contribution:}

\begin{itemize}
    \item Demonstrates a prototypical AI-enabled artefact capable of generating interpretable KPIs, clustering narrative inputs, and mapping initiatives to SDGs.
    \item Provides social enterprises and innovation intermediaries with a structured, semi-automated approach to enhance transparency, comparability, and evidence-informed decision-making.
\end{itemize}

\textbf{Methodological Contribution:}

\begin{itemize}
    \item Shows how Design Science Research can be applied to the development and evaluation of AI-supported artefacts in complex, value-driven domains.
    \item Highlights the importance of iterative evaluation cycles and human oversight in ensuring relevance and ethical alignment.
\end{itemize}

\section{Limitations}\label{sec:limitations}

\begin{itemize}
    \item The artefact is prototypical and not intended as a market-ready system; scalability, robustness, and long-term effects remain untested.
    \item Evaluation relied on synthetic and anonymized project data as well as a limited number of stakeholder walkthroughs.
    \item The current implementation is tailored to the \textit{Inluma} context and may require adaptation for other organizational or sectoral settings.
\end{itemize}

\section{Outlook and Future Work}\label{sec:outlook}

Future research and development may include:

\begin{itemize}
    \item Integration with larger datasets and live project pipelines to assess longitudinal impact development.
    \item Extension of AI capabilities for deeper qualitative analysis, such as narrative change over time, sentiment dynamics, or stakeholder perspective modeling.
    \item Adaptation of the artefact for broader application in public administration, social entrepreneurship ecosystems, and international development contexts.
    \item Further refinement of human-in-the-loop workflows to balance automation, transparency, and participatory decision-making.
\end{itemize}

\section{Closing Remarks}\label{sec:closing-remarks}

This thesis demonstrates that AI can act as a supportive cognitive tool in Impact Measurement and Management, enhancing sense-making and comparability while preserving human judgment and ethical oversight.
By integrating IMM frameworks, AI methods, and public value considerations, the proposed artefact offers a practical and reflective approach to evaluating social innovation initiatives in complex public and social contexts.



\printbibliography
\appendix
\chapter{Appendix A: Interview Guide}
\lipsum[17]

\chapter{Appendix B: Additional Data}
\lipsum[18]

\end{document}
