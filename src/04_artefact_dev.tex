\chapter{Artefact Development}\label{ch:artefact-development}

This chapter describes the design and implementation of the AI-supported Impact Measurement and Management (IMM) artefact for \textit{Inluma}.
It details the workflow from onboarding new projects, parsing and structuring pitch decks, AI-assisted KPI generation, and integration with the Public Value Academy platform.

\section{Project Onboarding and Pitch Deck Parsing}\label{sec:onboarding}

To reduce early-stage assessment pain points, a \textbf{Pitch Deck Parsing} function was developed:

\begin{itemize}
    \item PDF documents are processed using \texttt{PyPDF} to extract text and graphic information.
    \item AI models correct scrambled text, OCR errors, or formatting inconsistencies.
    \item Extracted data is structured with \texttt{Pydantic} classes for downstream processing.
\end{itemize}

\begin{figure}[H]
    \centering
    \begin{tikzpicture}[
        node distance=1.5cm,
        every node/.style={font=\sffamily, align=center},
        box/.style={rectangle, rounded corners, draw=black, fill=gray!10, minimum width=6cm, minimum height=1cm},
        arrow/.style={-{Stealth[length=3mm,width=2mm]}, thick}
    ]
    % Nodes
    \node[box] (upload) {Upload Pitch Deck (PDF)};
    \node[box, below=of upload] (pdf) {PDF Parsing \& Text Extraction (PyPDF)};
    \node[box, below=of pdf] (ai_clean) {AI-Enhanced Text Cleaning \& Scramble Correction};
    \node[box, below=of ai_clean] (struct) {Structured Data Mapping (Pydantic Classes)};
    \node[box, below=of struct] (profile) {Profile Creation: Company, Project, Impact Dimensions};
    \node[box, below=of profile] (output) {Output: Ready-to-Use Structured Project Profile};
    % Arrows
    \draw[arrow] (upload) -- (pdf);
    \draw[arrow] (pdf) -- (ai_clean);
    \draw[arrow] (ai_clean) -- (struct);
    \draw[arrow] (struct) -- (profile);
    \draw[arrow] (profile) -- (output);
    \end{tikzpicture}
    \caption{Automated Pitch Deck Parsing and AI-Enhanced Extraction Workflow (vertical layout).}
    \label{fig:pitchdeck-parsing}
\end{figure}

\subsection{Structured Project Profile}\label{subsec:profile-model}

The \texttt{Profile} Pydantic model captures essential company, founder, and project information:

\begin{verbatim}
class Profile(BaseModel):
    startup_name: str
    legal_form: str
    founder_first_name: str
    founder_last_name: str
    founder_gender: Gender
    startup_email: str
    startup_phone: str
    startup_city: str
    startup_country: str
    startup_postcode: str
    website: str
    project_beginning: str
    turnover: int
    profit: int
    employers: int
    problem: str
    vision: str
    mission: str
    solution: str
    social_impact: str
    reason: str
    value_1: str
    value_2: str
    value_3: str
    target_group: str
\end{verbatim}

\textbf{TODO:} Add example filled instance to illustrate real project onboarding.

\section{Indicator and KPI Generation}\label{sec:kpi-generation}

Following onboarding, the IMM phase begins:

\begin{itemize}
    \item Pre-generated library of over 1,600 indicators serves as reference.
    \item Function \texttt{gen\_k\_measurement\_kpi()} generates SMART KPIs for specific categories/subcategories.
    \item Each KPI contains short-term and long-term goals, measurement methods, units, survey questions, and justification.
    \item Optional secondary goals created if multiple outcomes are detected in input text.
\end{itemize}

\begin{figure}[H]
    \centering
    \begin{tikzpicture}[
        node distance=1.5cm,
        every node/.style={font=\sffamily, align=center},
        box/.style={rectangle, rounded corners, draw=black, fill=gray!10, minimum width=6cm, minimum height=1cm},
        arrow/.style={-{Stealth[length=3mm,width=2mm]}, thick}
    ]
    % Nodes
    \node[box] (input) {Pitch Deck / Project Profile};
    \node[box, below=of input] (parsing) {Text Parsing \& Cleaning (PyPDF + AI)};
    \node[box, below=of parsing] (structuring) {Structured Data Extraction (Pydantic)};
    \node[box, below=of structuring] (kpi) {KPI Generation via LLM (LangGraph)};
    \node[box, below=of kpi] (audit) {Audit \& Human-in-the-Loop Validation};
    \node[box, below=of audit] (sdg) {SDG \& Indicator Alignment};
    \node[box, below=of sdg] (output) {Output: Actionable KPIs \& Assessment Forms};
    % Arrows
    \draw[arrow] (input) -- (parsing);
    \draw[arrow] (parsing) -- (structuring);
    \draw[arrow] (structuring) -- (kpi);
    \draw[arrow] (kpi) -- (audit);
    \draw[arrow] (audit) -- (sdg);
    \draw[arrow] (sdg) -- (output);
    \end{tikzpicture}
    \caption{AI-Assisted KPI Generation Workflow (vertical layout for compact page fit).}
    \label{fig:kpi-generation}
\end{figure}

\begin{verbatim}
def gen_k_measurement_kpi(category: str, subcategory: str, k: int = 10):
    """
    Generate k distinct SMART IndicatorKPIs for a given category/subcategory using an LLM.
    """
    # Structured LLM output via API
    ...
\end{verbatim}

\textbf{TODO:} Include one full example of generated KPI in appendix with short-term and long-term indicators.

\section{Human-in-the-Loop Evaluation}\label{sec:hitl}

Generated KPIs and assessment forms are:

\begin{itemize}
    \item Reviewed by domain experts and stakeholders before deployment.
    \item Distributed to target groups for feedback and data collection.
    \item Iteratively refined for alignment with project objectives and public value principles.
\end{itemize}

\textbf{TODO:} Add example of human-in-the-loop feedback affecting KPI refinement.

\section{Integration with the Public Value Academy Platform}\label{sec:integration-platform}

\begin{itemize}
    \item Supports workshops and structured reflection around public value.
    \item Embeds human-in-the-loop feedback directly into workflows.
    \item Enables iterative improvement of AI-supported tools.
\end{itemize}

\section{Ethical and Governance Considerations}\label{sec:ethical-governance}

\begin{itemize}
    \item GDPR-compliant handling of participant and project data.
    \item Explainable AI (XAI) applied throughout parsing, KPI generation, and SDG mapping.
    \item Human oversight enforced in all critical stages.
\end{itemize}

\section{Next Steps and Data Analysis}\label{sec:next-steps}

\begin{itemize}
    \item \textbf{TODO:} Define the methodology for analyzing collected KPI and survey data, including:
        \begin{itemize}
            \item Aggregation and cleaning of responses from target groups,
            \item Statistical analysis for quantitative indicators,
            \item NLP or thematic analysis for qualitative inputs,
            \item Integration of findings with public value dimensions.
        \end{itemize}
    \item \textbf{TODO:} Determine thresholds or scoring rubric for KPI performance and public value metrics.
    \item \textbf{TODO:} Include mock-up or example of impact dashboard.
\end{itemize}

\begin{figure}[H]
\centering
\begin{tikzpicture}[
    node distance=1.5cm,
    box/.style={rectangle, draw=black, rounded corners, fill=gray!10, minimum width=8cm, minimum height=1cm, align=center},
    arrow/.style={->, thick}
]

% Nodes
\node[box] (pitchdeck) {Pitch Deck\\Parsing \& AI Extraction};
\node[box, below=of pitchdeck] (profile) {Startup Profile\\(Pydantic Model)};
\node[box, below=of profile] (kpi) {AI-Assisted\\KPI Generation};
\node[box, below=of kpi] (data) {Data Collection\\(Target Groups)};
\node[box, below=of data] (analysis) {Data Analysis\\Quantitative \& Qualitative \\ \textbf{TODO: Define methods}};
\node[box, below=of analysis] (dashboard) {Impact Dashboard\\for Stakeholders \\ \textbf{TODO: Mock-up / Metrics}};

% Arrows
\draw[arrow] (pitchdeck) -- (profile);
\draw[arrow] (profile) -- (kpi);
\draw[arrow] (kpi) -- (data);
\draw[arrow] (data) -- (analysis);
\draw[arrow] (analysis) -- (dashboard);

\end{tikzpicture}
\caption{End-to-End Vertical Workflow: From Pitch Deck Parsing to Impact Dashboard}
\label{fig:end-to-end-pipeline-vertical}
\end{figure}

\section{Summary}\label{sec:artefact-summary}

This chapter demonstrates that the AI-supported IMM artefact can:

\begin{itemize}
    \item Efficiently onboard new projects using automated pitch deck parsing.
    \item Generate structured project profiles with AI-assisted data extraction.
    \item Produce actionable KPIs aligned with SDGs and recognized impact frameworks.
    \item Maintain a human-in-the-loop workflow for quality assurance, interpretability, and stakeholder validation.
    \item Feed collected data into a dashboard for actionable insights for impact investors.
\end{itemize}

\textbf{TODO:} Fill in the analysis methods, dashboard design, and examples before final evaluation chapter.