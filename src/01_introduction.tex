%! Author = deandidion
%! Date = 09.07.25

\chapter{Introduction}
\section{Background}

Innovation in the public sector is increasingly seen as essential for tackling complex societal challenges. As governments and public institutions explore new ways to deliver services, assess policy outcomes, and engage with citizens, the question of \textbf{impact} becomes central. While private sector innovations often measure success through profit and efficiency, public sector innovation must be evaluated against broader societal value — a much more nuanced and multidimensional goal.

Artificial Intelligence (AI) has emerged as a powerful tool for analyzing large amounts of data, identifying patterns, and supporting decision-making. In the public sector, AI promises to improve transparency, accountability, and responsiveness. However, its potential role in \textbf{measuring the impact} of public innovation is still underexplored. Traditional impact measurement frameworks are often ill-suited for dynamic or experimental initiatives, and there is a growing need to develop more intelligent, adaptive approaches.

This thesis is embedded in a real-world initiative from the \textbf{Public Value Hub in Leipzig} and aligns with the goals of the \textbf{Public Value Academy}, an emerging digital platform aimed at fostering innovation literacy and sustainable impact measurement in the public sector.

\section{Problem Statement}
Despite the growing use of AI across sectors, there remains a significant gap in how AI can support \textbf{meaningful, qualitative impact assessment} — particularly in the public domain. Existing tools often rely on rigid indicators or retrospective analysis, failing to capture complexity, learning, or long-term public value creation. Furthermore, public sector organizations often lack the resources or knowledge to adopt and adapt AI tools effectively.

There is a need to explore \textbf{how AI technologies can be applied to support dynamic, context-sensitive, and participatory impact measurement}, integrating frameworks such as those developed by \textbf{PHINEO} and supported by \textbf{UnternehmerTUM’s educational content}.

\section{Objectives}
The aim of this thesis is to develop a \textbf{conceptual and technical framework} for AI-supported impact measurement in the public sector. By combining theory, stakeholder insights, and prototyping with Python-based methods, the goal is to investigate how such a system could function in practice — particularly as part of the Public Value Academy’s software platform.

\section{Research Questions}

This thesis is guided by the following research questions:

\begin{itemize}
\item
How can artificial intelligence contribute to improved impact measurement in public sector innovation?
\item
What are the challenges and opportunities of integrating AI with existing frameworks such as PHINEO’s IMM?
\item
What would a prototype AI-supported measurement tool look like in practice?
\end{itemize}

\section{Scope and Limitations}

This thesis focuses on the \textbf{conceptual design and prototype development} of an AI-supported measurement framework. The implementation is limited to \textbf{exploratory Python-based prototypes} and does not aim to develop a fully deployed system. While it draws from existing frameworks and stakeholder input, it does not include large-scale empirical validation.

The focus is on \textbf{public innovation projects} in the German context, though the framework could have broader applicability.


\section{Methodology Overview}

The research combines:
\begin{itemize}
\item
A literature review on impact measurement and AI in the public sector,
\item
Exploration of frameworks (such as PHINEO’s IMM),
\item
Qualitative insights from relevant stakeholders (e.g. Public Value Hub),
\item
And the development of basic Python-based prototypes to test technical feasibility and application logic.
\end{itemize}

\section{Structure of the Thesis}

This thesis is structured as follows:

\begin{itemize}
\item
\textbf{Chapter 2} presents the theoretical and conceptual background, including relevant literature on AI, impact measurement, and public value.
\item
\textbf{Chapter 3} outlines the methodology and design process used in this research.
\item
\textbf{Chapter 4} presents the key findings from prototype development and stakeholder insights.
\item
\textbf{Chapter 5} discusses the results in the context of existing frameworks and reflects on challenges and opportunities.
\item
\textbf{Chapter 6} concludes the thesis with a summary of key insights and recommendations for further development.
\end{itemize}