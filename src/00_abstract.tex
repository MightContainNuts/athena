%! Author = deandidion
%! Date = 11.07.25

% Document
\begin{abstract}
Measuring social and economic impact has become increasingly important in public sector innovation, as governments and organizations aim to demonstrate accountability and optimize resource use~\parencite{oecd2020, giin2023}.
The growing number of impact-focused startups and initiatives highlights a broader shift towards mission-driven work that prioritizes measurable public value and sustainable outcomes~\parencite{phineo2023, unternehmertum2023}.

This thesis explores how artificial intelligence (AI) can support impact measurement within this evolving context.
It is situated at the Public Value Hub in Leipzig and contributes to the ongoing development of the Public Value Academy software platform.
The primary goal is to develop a framework for AI-supported impact measurement that balances technical feasibility with alignment to public values.

The work draws on established Impact Measurement and Management (IMM) concepts from Phineo, alongside practical guidance from UnternehmerTUM. These inform both the conceptual foundations and the requirements for a scalable, real-world solution.

Methodologically, the research combines stakeholder interviews with the implementation of AI components in Python.
These components illustrate how intelligent systems can foster transparent, data-driven evaluations in a sector where social outcomes are paramount.

By bridging theory and practice, the thesis demonstrates how AI can responsibly enhance public innovation, promote accountability, and strengthen impact-oriented approaches in the public sector.
\end{abstract}