%! Author = deandidion
%! Date = 11.07.25

% Document
\begin{abstract}
Measuring social and economic impact has become increasingly important in public sector innovation, as organizations seek to demonstrate accountability, optimize resource use, and align their actions with public value objectives~\parencite{oecd2020, giin2023}.
Amid this development, artificial intelligence (AI) offers new possibilities for automating data analysis, improving transparency, and supporting evidence-based decision-making in Impact Measurement and Management (IMM).

This thesis applies the \textbf{Design Science Research (DSR)} methodology to design, develop, and evaluate an AI-supported IMM framework.
The research is situated within the \textit{Public Value Hub} in Leipzig and contributes to the development of the \textit{Inluma} platform for measuring and managing social impact.
Drawing on established IMM frameworks from Phineo and UnternehmerTUM, the work derives design requirements that integrate both technical feasibility and public value alignment.

The resulting artefact is implemented in a prototypical form as an initial instantiation and evaluated according to criteria of feasibility, usability, transparency, and comparability.
Through this iterative DSR process, the study bridges theory and practice, demonstrating how AI-driven approaches can responsibly enhance impact measurement and strengthen accountability in the public sector.

The expected contribution is threefold:
a scientifically grounded artefact design for AI-supported IMM,
a methodological illustration of DSR in the public innovation domain,
and practical insights for social enterprises and public organizations seeking to operationalize data-driven impact management.
\end{abstract}